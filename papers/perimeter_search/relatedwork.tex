\section{Related Work}
The work most closely related to ours is due to
\citeauthor{harabor10}~\shortcite{harabor10}. 
They introduce empty rectangular rooms,
a symmetry breaking technique specific to 4-connected grid maps
which we will refer to as 4ERR.
Main differences between 4ERR and our work are mentioned in 
Sections \ref{introduction} and \ref{rooms-based symmetry reduction}.
%Like our work, 4ERR decomposes maps into obstacle-free rectangles.
%We generalise this approach to 8-connected grid maps and introduce a new 
%offline node pruning algorithm that can significantly 
%reduce the number of nodes which need to be explored during search.
%We also give a new online pruning technique which can often reduce the
%branching factor associated with a given node and further improve search times.
%These enhancements make RSR more widely applicable than 4ERR and, 
%on the subset of problems to which both algorithms are suited, 
%RSR dominates the performance of 4ERR. 
%\par
%MSA* \cite{bolanca09} is another recent optimality preserving search algorithm which attempts to speed up search 
%on grid maps by exploiting path equivalence in empty rectangular rooms. 
%Rather than pruning nodes from the interior of an empty room however MSA* attempts to speed up 
%search by generating macro edges on the fly.
%An improvement over conventional A* is reported but the algorithm is also
%shown to expand a large number of nodes from the interior of empty rooms, which hampers its performance.
%It is also worth noting that MSA* uses a different empty room decomposition method
%from the one described in our work.
%\par
Our work is also related to that of \citeauthor{pochter10}~\shortcite{pochter10}. 
They introduce \emph{swamps}, an alternative search space reduction algorithm
which requires decomposing a graph into so called ``swamp'' areas that can be 
ignored because there always exists a symmetric path that does not cross any 
tiles in the swamp. 
Each search instance is then limited to an a corridor of inter-dependent 
swamp (and non-swamp areas) and all remaining nodes in the graph are ignored.
The identification of swamps is quite different to our empty room decomposition
and the focus of the algorithm appears to be in identifying regions of the
search space that can be ignored.
By comparison, RSR focuses on reducing the search effort required to explore
any given area.
%Additionally, swamps are shown to be most effective in areas featuring a large number of obstacles 
%and less effective on maps featuring wide open areas.
%By comparison, our algorithm is most effective when large empty rooms can be identified and less
%effective when this is not the case.
Thus the two methods are in some sense complementary.
\par
RSR bears some similarity to new heuristic methods aimed at improving the 
performance of standard A* on grid maps \cite{bjornsson06}.
In that work, like in ours, grid maps are decomposed into obstacle-free zones connected by entrances 
and exits.
A preliminary online search in the decomposed graph identifies zones that do not appear 
on any path between the start and goal node, thus yielding the \emph{dead-end heuristic}.
It can be seen as a technique for detecting areas that don't have to be searched
in the instance at hand.

Handling symmetry is a regularly appearing topic in the literature.
This is not surprising, given that,
in the presence of symmetry, search algorithms could be forced to evaluate 
many equivalent states, wasting time and making little real progress toward the
goal.
Dealing with symmetry is a regularly appearing topic in the literature. 
Symmetry has been addressed in planning \cite{},
constraint programming \cite{},
but also bin packing and operations research. 


%\par
%Contraction Hierarchies (CHs) \cite{geisberger08} are another recent technique 
%for reducing the size of the search space for the purpose of pathfinding.
%Originally developed in the context of road networks, CHs exploit the topology 
%of such graphs in order to create short-cut edges that allow a search algorithm to 
%reach the goal sooner.
%Though road networks are quite different to grid maps\footnote{Roadmaps tend to
%have a very high branching factor and often feature a small set of edges (representing freeways) 
%that frequently appear on many optimal paths between different nodes.} recent work has 
%shown that, on a popular grid-based benchmark, the technique can be up to six times faster, on average, 
%than standard A* search \cite{sturtevant10}.

% \par
% Fringe Search \cite{bjornsson05} is a general purpose graph search algorithm which also
% aims to improve on the performance of A*.
% This work is quite different from others we have discussed in that it does not
% rely on any specific decomposition technique nor on the development of any new heuristics
% to guide the search.
% Rather, it uses a novel iterative deepening technique that resumes each subsequent iteration from where the last
% iteration left off. 
% It is provably optimal if maximum search depth is sufficiently large and 
% it has been shown to run between 25-40\% faster than A*.
% As our work is predominately an offline graph-pruning technique, it could be combined with any search algorithm, including
% Fringe Search.
% \par
% Another effective method for solving path planning problems is to reformulate the original problem
% into an equivalent one in a much smaller abstract search space.
% Algorithms in this category are usually fast, memory-efficient and suboptimal.
% The HPA* algorithm \cite{botea04} uses a map decomposition approach,
% dividing a grid map into a series of fixed-size clusters connected 
% by entrances.
% As with Fringe Search, HPA* could be combined with our work on symmetry breaking.
% For example, first apply our pruning strategy and then apply HPA* to the resulting grid map.
