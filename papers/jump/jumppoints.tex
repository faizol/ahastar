\section{Jump Points}
In the previous section we developed rules for pruning neighbours during 
individual node expansions. We now extend this idea in order to define a 
macro-step operator which speeds up optimal search by identifying dead ends in the 
search space much sooner than would otherwise be the case.
The basic idea can be outlined as such:  following a pruning operation on some
arbitrary node $x$, we defer the immediate generation of each remaining
neighbour and, instead, continue travelling in the same direction as the
neighbour. At each step, we apply the appropriate pruning rules (straight or
diagonal) and continue this process until we reach an obstacle or identify an 
eventual successor which satisfies certain properties and which we term a 
\emph{jump point}.
If we find such a jump point, we generate it as a successor of $x$; if we do not 
find a jump point, we generate nothing.

\begin{definition}
\label{def:jump}
A node $n$ is designated a jump point if it satisfies at least one of the following
conditions:
\begin{itemize}
\item{$n$ is reachable by a straight (or diagonal) move but to which we
cannot apply Straight Pruning Rule 4 (Diagonal Pruning Rule 3).}
\item{$n$ is located on the same row or column of the grid as the goal.}
\item{$n$ has one or more neighbours which are themselves jump points.}
\end{itemize}
\end{definition}

\noindent
Figure \ref{fig:pruningrules}(b) shows an example of jump point which satisfies the
first condition of Defintion \ref{def:jump}: we cannot apply Straight Pruning
Rule 4 as the evaluation of neighbour 3 is forced.
The same is true of the example in Figure \ref{pruningrules}(d): we cannot
apply Diagonal Pruning Rule 3 as the evaluation of neighbour 1 is forced.


\subsection{Straight Jump Points}


\subsection{Diagonal Jump Points}



\subsection{Optimality}
Sketch: show that each pruning rule discards only neighbours
which cannot belong to the optimal path for the problem at hand.
Then, by induction, demonstrate that jump points have the same 
property. 

Complications:
 - jump points when passing the row or column associated with the goal
 - no jump points at the top of a row/column ending in an obstacle.

