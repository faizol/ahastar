\section{Related Work}
\label{sec:relatedwork}
 Approaches for identifying and eliminating search-space symmetry have been
proposed in areas including planning \cite{fox99}, constraint programming
\cite{gent00}, and combinatorial optimization \cite{fukunaga08}. 
Very few works however explicitly identify and deal with symmetry in pathfinding
domains such as grid maps. 

Empty Rectangular Rooms~\cite{harabor10} is an offline symmetry breaking technique
which attempts to redress this oversight. 
It decomposes grid maps into a series of obstacle-free rectangles and replaces all nodes 
from the interior of each rectangle with 
a set of macro edges that facilitate optimal travel. 
Specific to 4-connected maps, this approach is less general than
jump point pruning. It also requires offline pre-processing whereas our
method is online.

The \emph{dead-end heuristic} \cite{bjornsson06} and \emph{Swamps} \cite{pochter10}
are two similar pruning techniques related to our work.
Both decompose grid maps into a series of adjacent areas. Later, this decomposition
is used to identify areas not relevant to optimally solving a particular
pathfinding instance.
This objective is similar yet orthogonal to our work where
the aim is to reduce the effort required to explore any given area in the search
space.

A different method for pruning the search space is to identify \emph{dead} and
\emph{redundant} cells~\cite{sturtevant10}.  Developed in the context of
learning-based heuristic search, this method speeds up search only after running
multiple iterations of an iterative deepening algorithm.  Further, the
identification of redundant cells requires additional memory overheads which
jump points do not have.

\emph{Fast expansion}~\cite{sun09} is another related work that speeds up
optimal A* search. It avoids unnecessary open list operations when it finds a
successor node just as good (or better) than the best node in the open list.
Jump points are a similar yet fundamentally different idea: they allow us to
identify large sets of nodes that would be ordinarily expanded but which can be
skipped entirely.
 
In cases where optimality is not required, hierarchical pathfinding methods
are pervasive.  They improve performance by decomposing the
search space, usually offline, into a much smaller approximation.  Algorithms
of this type, such as HPA*~\cite{botea04}, are  fast and
memory-efficient but also suboptimal.
