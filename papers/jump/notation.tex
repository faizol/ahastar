\section{Notation and Terminology}
\label{sec:notation}
We work with undirected uniform-cost grid maps.  Each node has $\leq 8$
neighbours and is either traversable or not.  Each straight (i.e. horizontal or vertical) move,
from a traversable node to one of its neighbours, has a cost of 1; diagonal
moves cost $\approx\sqrt 2$.  Moves involving non-traversable (obstacle) nodes
are disallowed.  The notation $\vec{d}$ refers to one of the eight allowable
movement directions (up, down, left, right etc.).  We write $y = x + k\vec{d}$
when node~$y$ can be reached by taking $k$ unit moves from node~$x$ in direction
$\vec{d}$.  When $\vec{d}$ is a diagonal move, we denote the two
straight moves at $45\deg$ to $\vec{d}$ as $\vec{d_1}$ and
$\vec{d_2}$.

A path $\pi =~\begin{pth}n_{0}, n_{1}, \ldots , n_{k}\end{pth}$ is a cycle-free
ordered walk starting at node $n_{0}$ and ending at $n_{k}$.  We will sometimes
use the setminus operator in the context of a path: for example $\pi \setminus
x$. This means that the subtracted node $x$ does not appear on (i.e. is not
mentioned by) the path.  We will also use the function $len$ to refer the length
(or cost) of a path and the function $dist$ to refer to the distance between two
nodes on the grid: e.g. $len(\pi)$ or $dist(n_{0}, n_{k})$ respectively.  



