\subsection{Optimality}
In this section we prove that for each optimal length path in a grid map there
exists an equivalent length path which can be found by only expanding jump
point nodes during search (Theorem \ref{theorem:jumping}).  Our result is
derived by identifying for each optimal path a symmetric alternative which we
split into contiguous segments. We then prove that each \emph{turning point}
along this path is also a jump point.

\begin{definition}
\label{def:turningpoint}
A \emph{turning point} is any node $n_{i}$ along a path where the direction of
travel from the previous node $n_{i-1}$ to $n_{i}$ is different to the direction
of travel from $n_{i}$ to the subsequent node $n_{i+1}$.
\end{definition}

Before proving our main result, we will first develop an equivalence between
jump points and the turning points that appear along certain optimal length 
symmetric paths.
 
\begin{lemma}
\label{lemma:piprime}
Given an optimal length path $\pi$ on a grid map, it is always possible to
derive a symmetric path $\pi'$ along which every diagonal step is taken as
early as possible.
\end{lemma}
\begin{proof}
We proceed by examining pairs of adjacent edges appearing along $\pi$ where
where $(i, j)$ is a straight move and $(j, k)$ is a diagonal move.
The objective is to replace each such pair of edges with two new edges: 
$(i, j')$, which is a diagonal move and $(j', k)$ which is a straight move. 
The operation is successful if:
\begin{itemize}
\item $(i, j'$) and $(j', k)$ are both valid moves; i.e. node $j'$ is not an
obstacle.
\item $(i, j') + (j', k) = (i, j) + (j + k)$; i.e. the cost of
the path $\pi$ is unchanged by the operation.
\end{itemize}
We apply this procedure repeatedly until no further re-ordering of the moves
is possible and return $\pi'$, a new optimal path which is symmetric
to the original and along which diagonal moves appear as early as possible.
\end{proof}

\begin{lemma}
\label{lemma:turningpoints}
Each turning point of $\pi'$ is also a jump point.
\end{lemma}
\begin{proof}
Let $n_{k}$ be an arbitrary turning point node along $\pi'$. 
There are two cases to consider, depending on whether $n_{k}$ is next
to an obstacle or not.
\par
In the first case, the obstacle must be blocking of the four cardinal 
moves (``up'', ``down'', ``left'' or ``right''); otherwise we can re-write 
$\pi'$ s.t. $n_{k}$ is no longer mentioned.
If one of the four cardinal moves is blocked, $n_{k}$ has at least one
forced neighbour. This satisfies the second condition in Definition
\ref{def:jump} and we conclude $n_{k}$ is a jump point.
\par
In the second case $n_{k}$ is not adjacent to any obstacle but the edge
from its predecessor $n_{k-1}$ must be a diagonal step; otherwise we can 
re-write $\pi'$ s.t. $n_{k}$ is no longer mentioned.
Since $n_{k}$ is not adjacent to an obstacle, the next turning point along 
$\pi'$ must be; otherwise we can once more re-write $\pi'$ s.t $n_{k}$ is no longer
mentioned. Further, the next turning point along $\pi'$ must be reachable 
by a series of straight moves; otherwise $\pi'$ is not optimal.
This is sufficient to satisfy the third condition in Definition \ref{def:jump} and
we conclude $n_{k}$ is a jump point.
\end{proof}

\begin{theorem}
\label{theorem:jumping}
For each optimal length path between two nodes on a grid there exists
an equivalent length symmetric path that mentions only jump point nodes.
\end{theorem}
\begin{proof}
Let $\pi$ be an arbitrarily chosen optimal path between two nodes
on a grid.  
We begin by re-ordering $\pi$ according to Lemma \ref{lemma:piprime} and derive
a new symmetric path $\pi'$ along which diagonal steps are taken as early as 
possible.
\par
Next, we divide $\pi'$ into a series of adjacent segments s.t. 
$\pi' = \pi'_{0} + \pi'_{1} + \ldots + \pi'_{n} $. Each $\pi'_{i} = \lbrace n_{0}, n_{1},
\ldots, n_{k-1}, n_{k} \rbrace$ is a subpath along which all moves involve
travelling in the same direction (e.g.  only ``up'' or ``down'' etc).  Notice
that with the exception of the start and goal, every node at the beginning and
end of a segment is also a turning point.
\par
Since each $\pi'_{i}$ consists only of moves in a single direction
(straight or diagonal) we can use Algorithm \ref{alg:jump} to jump from $n_{0}
\in \pi'_{i}$, the node at beginning of each segment to $n_{k} \in \pi'_{i}$, the
node at the end, without necessarily stopping to expand every node in between.
Intermediate expansions may occur but the fact that we reach $n_{k}$
optimally from $n_{0}$ is guaranteed.
It remains to show only that both $n_{0}$ and $n_{k}$ are identified as
jump points and thus necessarily expanded. 
By Lemma \ref{lemma:turningpoints} each turning point along $\pi'$ is 
also a jump point, so every turning point node must be expanded during search.
Only the start and goal remain. These are jump points by definition and must
also be expanded.
\end{proof}
