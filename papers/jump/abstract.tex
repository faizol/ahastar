Pathfinding in a grid-based environment is a problem which commonly arises in
many application areas: for example robotics and video games.  The
state-of-the-art in this area is dominated by hierarchical pathfinding
algorithms which are fast and have small memory overheads but usually return
suboptimal paths.  
In this paper we present a novel search strategy, specific to grids, which is
fast, optimal and requires no memory overhead. Our algorithm can be described as
a macro operator which identifies and selectively expands only certain
nodes in a grid map which we call \emph{jump points}.  Intermediate nodes on a
path connecting two jump points are never expanded.  We prove  
that this approach always computes optimal solutions and then undertake
a thorough empirical analysis, comparing our method with other recent search
space reduction algorithms.  We find that searching with jump points can speed
up A* by an order of magnitude and more and report significant improvement over
the current state of the art.  We obtain these results on both synthetic and
real-world benchmarks taken from the literature, including two from the popular
video games \emph{Dragon Age: Origins} and \emph{Baldur's Gate II}.
