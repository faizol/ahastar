\section{Pruning Symmetric Paths Offline}
\label{algorithm}
Pathfinding in modern video games often involves exploring highly regular 
environments such as cities, sewers or dungeons (e.g Figure \ref{fig-bgmap}).
Though these locales tend to be topographically simple (usually being comprised
of empty rooms connected by corridors) they can also be highly symmetric 
with many optimal length paths existing between arbitrary pairs of locations.
Symmetry is undesirable as it increases the size of the search space
and forces search algorithms to waste time that could be spent finding better solutions.
\par
We propose the following offline strategy for identifying and eliminating symmetric paths in 
4-connected grid maps:
\begin{enumerate}
\item{Decompose the grid map into a set of empty rooms, where each empty room is 
rectangular in shape and free of any obstacles.}
\item{Prune all nodes from the interior of each empty room.}
\item{Add a series of \emph{macro edges} that connect each node on the perimeter of an empty room
with a node on the directly opposite side of the room.}
\end{enumerate}
Figure \ref{fig-overview} shows an example of this process.
It is easy see to see that we eliminate a large number of symmetric paths
that exist between arbitrary pairs of nodes on the perimeter of each room.
In many cases there remains only one optimal length path between each 
pair of nodes in each room.
We claim that this approach preserves optimality when traversing across any arbitrary room.

\begin{lemma}
\label{thm-roomtraversal}
Let $R$ be an arbitrary rectangular room that is free of obstacles
and $s, g \in R$ be two locations on its perimeter.
Then $s$ and $g$ can be connected optimally through a path that
mentions only nodes on the perimeter of $R$ and possibly involves
a macro edge.
\end{lemma}
\begin{proof}
\par
There are two distinct cases to consider.
Case 1 is when $s$ and $g$ are placed on the same side of the perimeter, or
on two orthogonal sides. 
To obtain an optimal path we can simply travel along the perimeter from $s$ to $g$.
Case 2 is when $s$ and $g$ are placed on opposite sides of the perimeter.
To obtain an optimal path we can simply follow the macro edge at $s$ 
to navigate directly to a node $s'$ located on
the same side of the perimeter as $g$. Then, go from $s'$ to $g$ along the perimeter.
\end{proof}

A direct corrolary to Lemma \ref{thm-roomtraversal} is that we can prune from consideration
all nodes from the interior of $R$ and limit ourselves to only searching nodes appearing along its perimeter.
