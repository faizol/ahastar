\section{Conclusion}
We presented Optimal Hierarchical A* (OHA*); a new optimality preserving algorithm for 
computing shortest paths on 4-connected grid maps.
We use a hierarchical planning approach that decomposes a map into empty rectangular rooms.
We prove that it is possible to optimally navigate across such rooms by only ever
expanding nodes on their perimeter and never expanding any from the interior.
We undertake an empirical analysis on a range of realistic benchmarks and show that
this approach is between 1.7 to 3.3 times faster than A*.
It is not hard however to find examples of maps that feature larger empty areas
\footnote{For example, such as those seen in Blizzard's popular multi-player game World of Warcraft} 
on which the performance of OHA* would be many more times faster. 
In addition to being fast OHA* is also shown to be very memory efficient,
expanding between 40-70\% fewer nodes than A* on the same set of realistic benchmarks.
Finally, our method is orthogonal to existing techniques meaning it could be easily integrated
as part of a larger framework involving specialised heuristics or other speedup techniques; 
for example as described in \cite{bjornsson05,bjornsson06}. 
\par
The obvious direction for further work is extending the algorithm to optimally solving 
shortest path problems in more popular 8-connected grid maps. 
One approach involving the use of macro successors has already been tried 
\cite{bolanca09} but we believe a different method that identifies 
dominated paths in fixed-size non-empty rooms could be promising. 
Another direction for future work is to develop decomposition algorithms that identify maximally
sized empty areas on a given grid map.
We believe this would significantly improve the performance of the current method.
