\section{Results}
\label{sec-results}
We evaluated the effectiveness of our graph pruning algorithm by measuring the 
total number of nodes pruned from each map (Table \ref{table-graphsize}) and
the average speedup experienced by A* in terms of nodes expanded and search time
(Figure \ref{fig-speedup}).
We measure the latter by sorting all experiments from each benchmark into
groups based on path length. 
The first group contains of experiments where the optimal path had a length
of between 1-25. 
Experiments in the second group have path lengths of between 26-50 and so on.
%This approach allows for a more detailed analysis than simply measuring average
%performance across an entire benchmark.
We will discuss our results on each benchmark in turn.
\input graphsize
\textbf{Adaptive Depth:} 
The topography of the maps in this benchmark were very favourable for our
symmetry breaking technique.
Our decomposition algorithm was able to identify many large open areas and
pruned between 50.5\% to 62.3\% of all nodes.
Its average performance was just over over 57\%. 
We also noticed a 2-3 fold reduction in the average number of nodes expanded 
by A* and a corresponding search time speed up of between 3-4 times.
Very easy problems (those with path lengths < 25) fared worse than harder
problems; there was an average speedup of 3 vs. 3.5 to 4 for harder problems.
This is as expected; the number of pruned nodes which do not need to be evaluated
will usually grow with the distance from the starting position to the goal
position. 
Of course this observation assumes the decomposition algorithm is able identify
symmetric paths in all rooms that appear on the way to the goal.
Though this is often the case there is no guarantee in general and the performance
gain experienced by the search algorithm is closely tied to the topography of 
the map.
\par
\textbf{Baldur's Gate: }
The maps in this benchmark are usually composed of large open areas, sometimes
interspersed with large obstacles, or otherwise small to medium rooms connected
by long rectangular corridors.
We expected to see similar performance to that observed on Adaptive Depth
however this was not the case.
Though our decomposition algorithm prunes as many as 70\% of all traversable nodes 
on some maps its average performance was only 29\%. 
There was also a reasonably high level variability in the effectiveness of the 
decomposition from one map to the next, as indicated by the standard deviation
of 10.33\%.
Some closer investigation revealed that the 45-degree orientation of these maps
resulted in long sequences of increasing clearance values running left-to-right.
This causes our decomposition algorithm to terminate early and results in long
skinny rooms which have few interior nodes.
Nevertheless, we observed that the number of nodes expanded by A* decreased by a
factor of between 1.5 to 1.8 corresponding to a search time speed up of between
2 to 2.5 times faster than A* running on the original map.
We expect that these results would improve significantly given a more effective
decomposition algorithm.
\par
\textbf{Rooms:}
The maps in this benchmark were all very similar, comprising of 32$\times$32
rectangular rooms connected by randomly placed entrances.
Each room is of size 7$\times$7 and contains 49 nodes. 
24 of these (or just under 50\%) are interior nodes which we expected would be
pruned.
This was indeed the case: between 46.4\% to 48.1\% of nodes were pruned by 
the decomposition algorithm with an average performance of 47.55\%.
We observed a similar reduction in the number of nodes expanded by A* yielding
search times approximately 2.75 times faster than A* running on the original
maps.
Given rooms with proportionally larger dimensions we would expect to see a
proportionally larger improvement in the performance of A*.
We expect the same is also true as rooms become smaller: in the worst case,
there are no interior nodes to prune and the performance of A* would remain
unchanged when compared to the original map.
