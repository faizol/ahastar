\section{Offline Symmetry Elimination}
\label{algorithm}
Pathfinding in modern video games often involves exploring highly regular 
environments such as cities, sewers or dungeons (e.g Figure \ref{fig-bgmap}).
Though these locales tend to be topographically simple (usually being comprised
of empty rooms connected by corridors) they can also be highly symmetric 
with many optimal length paths existing between arbitrary pairs of locations.
Symmetry appears in many domains (e.g. constraint programming \cite{walsh07})
and, unless it is handled properly, almost always increases the size of the search space
and forces search algorithms to waste time.
\par
We propose the following offline strategy for identifying and eliminating symmetric paths in 
4-connected grid maps:
\begin{enumerate}
\item{Decompose the grid map into a set of empty rooms, where each empty room is 
rectangular in shape and free of any obstacles. 
The size of the rooms can vary across a map, depending
on the placement of the obstacles.}
\item{Prune all nodes from the interior but not the perimeter of each empty
room.}
\item{Add a series of \emph{macro edges} that connect each node on the perimeter of an empty room
with a node on the directly opposite side of the room 
\footnote{Alternatively, macro edges could be generated on-the-fly during search. 
This obviates the need to store them explicitly.}.
The cost of each edge is equal to the Manhattan distance between its two endpoints.
}
\end{enumerate}
Trivial rooms which contain no interior nodes (for example rooms with a width $w$ or height $h$ 
$\leq 2$) are left unmodified by steps 2 and 3.
Figure \ref{fig-overview} shows an example of this process.
For each non-trivial room we prune $(w-2)\times(h-2)$ interior
nodes and, in the proceess, eliminate a large number of symmetric paths between 
nodes on the perimeter.
We claim that this approach preserves optimality when traversing across any arbitrary room.

\begin{lemma}
\label{thm-roomtraversal}
Let $R$ be an arbitrary rectangular room that is free of obstacles
and $m, n$ be two locations on its perimeter.
Then $m$ and $n$ can be connected optimally through a path that
mentions only nodes on the perimeter of $R$ and possibly involves
a macro edge.
\end{lemma}
\begin{proof}
\par
There are two distinct cases to consider.
Case 1 is when $m$ and $n$ are placed on the same side of the perimeter, or
on two orthogonal sides. 
To obtain an optimal path we can simply travel along the perimeter from $m$ to $n$.
Case 2 is when $m$ and $n$ are placed on opposite sides of the perimeter.
To obtain an optimal path we can simply follow the macro edge at $m$ 
and navigate directly to a node $m'$ located on
the same side of the perimeter as $n$. Then, go from $m'$ to $n$ along the perimeter.
The resultant path is optimal as its length is equal to the Manhattan distance between $m$ and $n$.
\end{proof}

A direct corollary to Lemma \ref{thm-roomtraversal} is that we can prune from consideration
all nodes from the interior of $R$ and limit ourselves to only searching nodes appearing along its perimeter. 
The only remaining consideration is how to deal with interior nodes that happen
to be the start or goal location for the search at hand.
We address this case in the following section.
