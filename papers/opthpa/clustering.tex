\section{Identifying Empty Rooms}
In this section we give a simple but effective algorithm for decomposing a grid map
into large empty rooms.
We proceed by first associating with every tile on the map a \emph{clearance value} 
that measures the amount of traversable space at each location.
We adopt the computation technique described in \cite{harabor08} where clearances
are used to facilitate pathfinding for multi-size agents navigating across grid maps.
The approach is a straightforward one which we outline in Algorithm \ref{alg_clearance}.
Note that our method is more efficient than the equivalent technique presented
in \cite{harabor08}.

\input alg_clearance

Once we derive a set of clearance values we employ a simple flood-fill algorithm to 
construct empty rooms:

\begin{enumerate}
\item{Iterate over all nodes on the map, identifying nodes that have not yet 
been assigned to a room.}
\item{Starting with an unassigned node attempt a maximal extension of the room
by flood-filling over the nodes in the current row (to the right of the start node).}
\item{Stop flood-filling when one of the following conditions is met: an obstacle is found,
a node which is already assigned to another room is found or an unassigned
node is found which has a larger clearance value than the node at the origin of the room.
The length of the row of tiles prior to termination is the width of the room.}
\item{Repeat Steps 2-3 for each subsequent row attempting to grow the room
each time to its maximal width.
The number of rows successfully processed before the algorithm terminates is the height of the room.}
\end{enumerate}

We found this algorithm to work significantly better than other more naive flood-fill
approaches (for example, as described in \cite{bjornsson06}). 
In particular, the early termination condition at Step 3 
due to increasing clearance is critical to the effectiveness of our approach.
We use clearances in this context to avoid extending several smaller rooms into an area which
could be occupied by one large room.

As we will show in Section \ref{sec-results} the performance of OHA* is closely
tied with its ability to prune as many nodes as possible from the interior of 
empty rooms. 
Although our method is not optimal for this purpose it is simple
to understand and implement and produces good results in practice.
