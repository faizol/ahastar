\section{Introduction}
Most path planning studies have been focused in one of two directions: identifying better heuristics to guide search and reducing the search space through hierarchical decomposition.
In the case of the former, it has been shown (by Malte \& Roger in AAAI-08) that even almost perfect heuristics can result in poor performance. 
In the case of the latter, the search space is reduced but optimality is not preserved. 
It seems then that the reduction of search spaces while preserving optimality, as we do in this work, is a good direction for research.
\par
We consider the problem of optimal path planning on 4-connected grid maps; 
so called because each tile in the grid can have at most 4 neighbours (one in each of the cardinal directions).
This simple domain appears in application areas such as robotics \cite{latombe91} and video games 
(cite one of the Game/AI Programming Gems books).
Although not as popular as 8-connected grid maps (which allow diagonal transitions) the 4-connected case has 
the advantage of producing many highly symmetrical paths; 
a property we exploit in order to derive the main result of this paper.
\par 
Our work is motivated by the seemingly innocuous problem in Figure \ref{fig-emptymap} 
which requires finding a shortest path from S to G in a mostly empty room.
This scenario appears as a subproblem in many real-world path planning applications;
for example, video games such as  BioWare's \emph{Neverwinter Nights} (cite?) feature complex dungeon
 areas that are composed of adjacent mostly empty rooms.
If we apply A* \cite{hart68} to solving the instance in Figure \ref{fig-emptymap}, even usinng a perfect heuristic, we find that the algorithm must expand all tiles in the grey area and at least some of those in the 
white area.
Given an unlucky tie-breaking strategy A* will generate all nodes on the map and of those expand all but one.
\par
In solving such problems it is helpful to observe that although many optimal length solutions exist
all are dominated by the two paths which may be found along the perimeter of the room. 
We generalise this observation to the related problem of traversing across empty rooms and show that
it is possible to optimally navigate across such areas without ever exploring tiles in the interior.
We undertake an empirical analysis and show that our technique retains the optimality guarantees of A* 
yet performs similarly to the state of the art memory-efficient but suboptimal hierarchical search methods.

