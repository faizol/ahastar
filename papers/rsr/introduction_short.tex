\section{Introduction}
\label{sec:introduction}
Pathfinding on undirected uniform-cost grid maps is a problem commonly appearing in
the literature: for example in application areas such as robotics \cite{lee09}
artificial intelligence \cite{wang09} and video games \cite{davis00,sturtevant10}.  
To solve pathfinding problems quickly practitioners usually apply
%In such contexts it is often the
%case that queries sent to the pathfinding system need to be solved as quickly as
%possible.  Traditionally, this requirement is met through the application of
hierarchical decomposition techniques such as HPA* \cite{botea04,sturtevant10} or 
develop more accurate heuristics to guide search~\cite{bjornsson06,sturtevant09,goldenberg10}.  
Each of these has disadvantages: either the returned solutions are
not guaranteed optimal or a substantial memory overhead is incurred.
\par
In this paper we present Rectangular Symmetry Reduction (RSR): a graph pruning
algorithm for undirected uniform-cost grid maps which is fast, memory efficient,
optimality preserving and which can, in some cases, eliminate entirely the need
to search.  
The central idea that we will explore involves the identification
and elimination of path symmetries from the search space. 
Along the way we generalise a pruning technique proposed in \cite{harabor10} 
which is only applicable to 4-connected uniform cost grid maps.
\par
When compared to other pruning methods from the literature we
find RSR has complementary strengths. We identify classes of instances
where RSR is clearly the better choice and show that it can often dominate convincingly
across a variety of synthetic and realistic benchmarks.

%When compared to \cite{harabor10}, we find that RSR extends the applicability 
%and improve the performance  of the earlier method.
%We also find that RSR has complementary strengths compared to both Swamp-based
%pathfinding~\cite{pochter10} and the enhanced Portal Heuristic
%\cite{goldenberg10}. We then identify classes of instances where RSR is clearly
%the better choice, and show that it dominates convincingly across a large number
%of instances.
%
