\section{Introduction}
\label{sec:introduction}
Pathfinding systems that operate on uniform-cost grid maps are common in the AI
literature and often used in
application areas such as robotics and real-time
video games.  Typical speed-up enhancements in such contexts include reducing
the size of the search space using abstraction \cite{botea04} and developing new
heuristics to more accurately guide search toward the goal \cite{sturtevant09}.
Though effective each of these strategies has shortcomings.  For example,
abstraction methods usually trade optimality for speed.  Meanwhile, improved
heuristics usually require significant extra memory.
\par 
My research proposes a new technique for speeding up grid-based pathfinding.  I
work specifically on the problem of identifying and eliminating symmetric path
segments from the search space.  To achieve this, I decompose an arbitrary grid
map into a set of empty rectangles and remove from each rectangle all interior
nodes and possibly some from along the perimeter.  A series of macro edges are
then added between selected pairs of remaining nodes in order to facilitate
provably optimal traversal through each rectangle.  The new algorithm,
Rectangular Symmetry Reduction (RSR), can speed up A* search by up to 38
times on a range of uniform cost maps taken from the literature.  In addition to
being fast and optimal, RSR requires no significant extra memory and is largely
orthogonal all existing speedup techniques.  When compared to the state of the
art, RSR often shows significant improvement across a range of benchmarks.
