\section{Future Work}
\label{sec:futurework}
Several directions are available for future work:
\begin{itemize}
\item{\textbf{Better decomposition methods:} the performance of RSR depends on the 
topography of individual maps. In the presence of large rooms or wide open 
areas RSR can often compute optimal paths much faster than searching on the original
map. On less favourable map topographies we achieve more modest improvements.
I would therefore like to explore alternative decomposition techniques, based
for example on convex shapes, which would allow bigger empty regions to be
identified and lead to better performance.}
\item{\textbf{Synthesis with other pruning methods:} RSR is orthogonal to almost
all existing techniques for speeding up pathfinding. It could therefore be
integrated as part of a larger framework involving specialised heuristics or
other graph pruning and state space reduction techniques; for example as
described in \cite{botea04,bjornsson05,bjornsson06}.}
\item{\textbf{Stronger online pruning techniques:} I am currently investigating
alternative pruning techniques which can be applied online and combined with RSR.
}
\end{itemize}
