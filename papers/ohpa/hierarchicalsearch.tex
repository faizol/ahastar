\section{Hierarchical Search}
\label{ia-sec:hierarchicalsearch}
We use a variation of the A* algorithm (cite Hart  et al) which involves finding all non-dominated paths from the start to the goal.

\subsection{Inserting the start and goal}
We begin by adding two temporary nodes into the abstract graph to represent the start and goal.
To preserve optimality, both the start and goal node need to be connected to each node that belongs to every entrance in the start and goal clusters. 
We use Dijkstra's algorithm to achieve this step and compute intervals for the upper and lower-bound traversal distances from the start and goal to each entrance.

\subsection{Searching in the graph}
We use a variation on the standard A* algorithm to find a solution. 
One difference is that we order nodes on the open list by using the upper-bound traversal cost associated with the edge connecting the two neighbours. 
Another difference is that we do not stop when an hierarchical solution is found. 
Instead, we keep expanding nodes until the open list contains no mode non-dominated paths.
A path $a$ is dominated by another path $b$ iff: $g_{a} > G_{b}$; ie. the lower-bound cost of $a$ is more than the upper-bound cost on $b$.
We need some examples of overlapping intervals to show situations in which one path dominates another and situations in which two paths do not dominate each other
\par
The optimal path is guranteed to lie along one of the non-dominated hierarchical paths (do we need a short proof of this statement or is it obvious?).
