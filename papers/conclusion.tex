\section{Conclusion} 
Heterogeneity in path planning is characteristic of many real-world problems but has received very little attention to date.
In this paper we have addressed this issue by showing how clearance-based obstacle distances can be computed and leveraged to improve path planning for multi-size agents in heterogeneous-terrain grid-world environments. 
Our approach reduces complex problems involving agents of different sizes and multi-terrain traversal capabilities to much simpler single-size, single-terrain search problems.
Building on these new insights, we have introduced a new planner, Annotated Hierarchical A*,  and have shown through comparative analysis that AHA* is able to find near-optimal solutions to problems in a wide range of environments yet still maintain exponentially lower search effort over standard A*.
Our hierarchical abstraction technique is simple to apply but very effective; we have shown that in most cases the overhead for storing the abstract graph is a small fraction of that associated with non-abstract graphs.
\par \indent
Future work could involve looking at computing annotations to deal with elevation and other common terrain features. 
We are also interested in finding a better inter-edge placement approach and reducing the effort to insert the start and goal into the abstract graph.
Finally, we believe AHA* could be usefully applied to solving heterogeneous multi-agent problems.
